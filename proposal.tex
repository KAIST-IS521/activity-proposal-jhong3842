\documentclass[a4paper, 11pt]{article}

\usepackage{kotex} % Comment this out if you are not using Hangul
\usepackage{fullpage}
\usepackage{hyperref}
\usepackage{amsthm}
\usepackage[numbers,sort&compress]{natbib}

\theoremstyle{definition}
\newtheorem{exercise}{Exercise}

\begin{document}
%%% Header starts
\noindent{\large\textbf{IS-521 Activity Proposal}\hfill
                \textbf{김재홍}} \\
         {\phantom{} \hfill \textbf{jhong3842}} \\
         {\phantom{} \hfill Due Date: April 15, 2017} \\
%%% Header ends

\section{Activity Overview}

Activity-proposal로 shellcode 제작에관해서 제안 \\

악성코드 분석과 exploit 분석시 shellcode를 만나게 되는 경우 발생\\

이때 쉘코드를 만들어 보았으면, 이러한 상황을 겪었을 때 좀 더 편하게 진행 가능\\

이를 통해서 shellcode가 무엇이고, 어떻게 동작 되는지 등. 시스템에 관한 이해를 도음\\

little endian, stack, syscall, 표준 입력 함수 특징 등에 관한 이해

\section{Exercises}

\begin{exercise}

  첫 번째로 기본적인 어셈블리 언어를 이용,syscall(write)를 통해서 자신의 github id를 출력합니다.

\end{exercise}

\begin{exercise}

  두 번째로 기본적인 쉘코드인 execve("/bin/sh",argv,null)과 같이 쉘을 뛰우는 어셈블리어 코딩을 해봅니다.


\end{exercise}

\begin{exercise}

  세 번째로 두 번째 진행되었던 기계어를 가지고 실제 사용할수 있게하기 위해서 바이트수를 제한, null 값 제거 등 현실 상황에 맞게 수정합니다.

\end{exercise}

\section{Expected Solutions}

  첫번째 단계에서는 출력하는 어셈블리어 코드를 제출합니다.\\
  두번째 단계에서는 쉘을 뛰우는 어셈블리어 코드를 제출합니다.\\
  세번째 단계에서는 해당 쉘코드를 가지고 실행하는 Makefile과 소스코드를 제출합니다.\\


세번째 단계 예)\\

   소스코드\\
	char* shellcode = "자신이 만든 쉘코드";\\
	
	void (*shell)(void);\\

	shell = (void*)shellcode;\\

	printf("쉘코드 길이 출력");
	shell();
	return 0;
  위 파일을 make하고 실행 후 체점 
 

\end{document}
